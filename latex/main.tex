\documentclass[12pt,a4paper]{article}
\usepackage[utf8]{inputenc}
\usepackage{geometry}
\usepackage{amsmath}
\usepackage{amsfonts}
\usepackage{amssymb}
\usepackage{graphicx}
\usepackage{float}
\usepackage[
    pdftitle={Laporan Penelitian - Productivity System},
    pdfauthor={Nama Mahasiswa},
    pdfsubject={Machine Learning Research},
    pdfkeywords={machine learning, research, productivity},
    colorlinks=true,
    urlcolor=blue,
    linkcolor=black,
    citecolor=blue
]{hyperref}
\usepackage{setspace}
\usepackage{fancyhdr}
\usepackage{enumitem}
\usepackage{listings}
\usepackage{xcolor}
\usepackage{booktabs}
\usepackage{array}
\usepackage{lipsum}
\usepackage[backend=biber,style=ieee]{biblatex}
\addbibresource{references.bib}

\lstset{
    basicstyle=\ttfamily\footnotesize,
    breaklines=true,
    frame=single,
    language=Python,
    numbers=left,
    numberstyle=\tiny,
    backgroundcolor=\color{gray!10},
    commentstyle=\color{green!50!black},
    keywordstyle=\color{blue},
    stringstyle=\color{red}
}

\geometry{margin=2.5cm}
\onehalfspacing

\pagestyle{fancy}
\fancyhf{}
\cfoot{\thepage}
\renewcommand{\headrulewidth}{0pt}

\title{Judul Laporan}
\author{Nama Mahasiswa\\
        NIM: 123456789\\
        Kelas: X}
\date{\today}

\begin{document}

\maketitle
\thispagestyle{empty}

% \newpage
% \tableofcontents
% \newpage

\section{Pendahuluan}
\subsection{Latar Belakang}
\lipsum[1-2]

\subsection{Rumusan Masalah}
Berdasarkan latar belakang yang telah dijelaskan, dapat dirumuskan masalah sebagai berikut:
\begin{enumerate}
    \item Bagaimana cara mengimplementasikan algoritma machine learning?
    \item Apa saja tantangan dalam pengembangan sistem ini?
    \item Bagaimana evaluasi performa sistem yang telah dibuat?
\end{enumerate}

\subsection{Tujuan}
Tujuan dari penelitian ini adalah:
\begin{itemize}
    \item Mengembangkan sistem yang efisien dan akurat
    \item Melakukan analisis performa terhadap berbagai metode
    \item Memberikan rekomendasi untuk pengembangan lebih lanjut
\end{itemize}

\section{Tinjauan Pustaka}
\subsection{Konsep Dasar}
\lipsum[3]

\subsection{Penelitian Terkait}
Berbagai penelitian telah dilakukan dalam bidang machine learning \cite{smith2023machine}. Algoritma klasik seperti yang dijelaskan oleh \cite{brown2022algorithms} masih relevan dalam konteks modern. Pendekatan deep learning yang dikembangkan oleh \cite{wilson2023deep} menunjukkan hasil yang menjanjikan.

\subsubsection{Metode Klasik}
\lipsum[5]

\subsubsection{Pendekatan Modern}
\lipsum[6]

\section{Metodologi}
\subsection{Desain Penelitian}
\lipsum[7]

\subsection{Dataset dan Preprocessing}
Dataset yang digunakan dalam penelitian ini terdiri dari beberapa komponen seperti yang ditunjukkan pada Tabel \ref{tab:dataset}.

\begin{table}[H]
\centering
\caption{Karakteristik Dataset}
\label{tab:dataset}
\begin{tabular}{@{}lcc@{}}
\toprule
\textbf{Kategori} & \textbf{Jumlah} & \textbf{Persentase} \\
\midrule
Training Data & 8,000 & 80\% \\
Validation Data & 1,000 & 10\% \\
Test Data & 1,000 & 10\% \\
\bottomrule
\end{tabular}
\end{table}

\subsection{Implementasi Algoritma}
Berikut adalah contoh implementasi algoritma yang digunakan:

\begin{lstlisting}[caption=Implementasi Algoritma Machine Learning, label=lst:algorithm]
import numpy as np
import pandas as pd
from sklearn.model_selection import train_test_split
from sklearn.ensemble import RandomForestClassifier
from sklearn.metrics import accuracy_score, classification_report

def load_and_preprocess_data(file_path):
    """
    Memuat dan melakukan preprocessing data
    """
    data = pd.read_csv(file_path)

    # Membersihkan data dari nilai null
    data = data.dropna()

    # Normalisasi fitur numerik
    numeric_features = data.select_dtypes(include=[np.number]).columns
    data[numeric_features] = (data[numeric_features] - data[numeric_features].mean()) / data[numeric_features].std()

    return data

def train_model(X_train, y_train):
    """
    Melatih model Random Forest
    """
    model = RandomForestClassifier(
        n_estimators=100,
        max_depth=10,
        random_state=42
    )

    model.fit(X_train, y_train)
    return model

# Main execution
if __name__ == "__main__":
    # Load data
    data = load_and_preprocess_data("dataset.csv")

    # Split features and target
    X = data.drop('target', axis=1)
    y = data['target']

    # Train-test split
    X_train, X_test, y_train, y_test = train_test_split(
        X, y, test_size=0.2, random_state=42
    )

    # Train model
    model = train_model(X_train, y_train)

    # Evaluate
    y_pred = model.predict(X_test)
    accuracy = accuracy_score(y_test, y_pred)

    print(f"Accuracy: {accuracy:.4f}")
    print(classification_report(y_test, y_pred))
\end{lstlisting}

\section{Hasil dan Pembahasan}
\subsection{Hasil Eksperimen}
\lipsum[8]

Hasil evaluasi performa model ditunjukkan pada Tabel \ref{tab:results}.

\begin{table}[H]
\centering
\caption{Hasil Evaluasi Model}
\label{tab:results}
\begin{tabular}{@{}lccc@{}}
\toprule
\textbf{Model} & \textbf{Accuracy} & \textbf{Precision} & \textbf{Recall} \\
\midrule
Random Forest & 0.8945 & 0.8712 & 0.9123 \\
SVM & 0.8734 & 0.8456 & 0.8934 \\
Neural Network & 0.9012 & 0.8823 & 0.9245 \\
\bottomrule
\end{tabular}
\end{table}

\subsection{Analisis Performa}
\lipsum[9]

\subsubsection{Perbandingan Metode}
\lipsum[10]

\subsubsection{Validasi Hasil}
Validasi dilakukan menggunakan k-fold cross validation dengan k=5. Hasil menunjukkan konsistensi performa yang baik across different folds.

\begin{figure}[H]
\centering
\includegraphics[width=1.0\textwidth]{system_architecture.png}
\caption{System Architecture Diagram}
\label{fig:architecture}
\end{figure}

\begin{figure}[H]
\centering
\fbox{\begin{minipage}{0.7\textwidth}
\centering
\vspace{2cm}
[Placeholder untuk grafik hasil validasi]
\vspace{2cm}
\end{minipage}}
\caption{Grafik Hasil Validasi Cross-Validation}
\label{fig:validation}
\end{figure}

\subsection{Diskusi}
\lipsum[11]

\section{Kesimpulan}
\subsection{Ringkasan Penelitian}
\lipsum[12]

\subsection{Kontribusi}
Kontribusi utama dari penelitian ini meliputi:
\begin{enumerate}
    \item Pengembangan framework yang dapat digunakan untuk kasus serupa
    \item Analisis komprehensif terhadap berbagai metode yang ada
    \item Rekomendasi praktis untuk implementasi di dunia nyata
\end{enumerate}

\subsection{Keterbatasan dan Saran}
\lipsum[13]

\printbibliography[title=Daftar Pustaka]

\end{document}